%%%%%%%%%%%%%%%%%%%%%%foreword.tex%%%%%%%%%%%%%%%%%%%%%%%%%%%%%%%

\seriesforeword

%Uses svmono with the hyperref package
%To be used with ALL main series volumes
\vspace{-1.0cm}
Dear reader,

Scientific research is increasingly interdisciplinary, and both students and experienced researchers often face the need to learn the foundations, tools, and methods of a new research field. This process can be quite demanding, and typically involves extensive literature searches and reading dozens of scientific papers in which the notation and style of presentation varies considerably. Since the establishment of this series in 2016 by founding editor-in-chief Aslak Tveito, the briefs in this series have aimed to ease the process by introducing and explaining important concepts and theories in a relatively narrow field, and to outline open research challenges and pose critical questions on the fundamentals of that field. The goal is to provide the necessary understanding and background knowledge and to motivate further studies of the relevant scientific literature. A typical brief in this series should be around 100 pages and should be well suited as material for a research seminar in a well-defined and limited area of computing. 

We publish all items in this series under the SpringerOpen framework, as this allows authors to use the series to publish an initial version of their manuscript that could subsequently evolve into a full-scale book on a broader theme. Since the briefs are freely available online, the authors do not receive any direct income from the sales; however, remuneration is provided for every completed manuscript. Briefs are written on the basis of an invitation from a member of the editorial board. Suggestions for possible topics are most welcome and can be sent to sundnes@simula.no. 

\bigskip

%\noindent March 2023 \hfill The editorial board:

\begin{flushright}
\leftline{March 2023 \hfill  \textit{Dr.\ Joakim Sundnes}}
\vspace{-5pt}
Editor-in-Chief\\
Simula Research Laboratory\\
sundnes@simula.no \\
\vspace{10pt}
\textit{Dr.\ Martin Peters} \\
Executive Editor Mathematics\\
Springer Heidelberg, Germany\\
martin.peters@springer.com\\
\vspace{3pt}


\end{flushright}