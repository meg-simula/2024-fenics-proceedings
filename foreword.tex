%%%%%%%%%%%%%%%%%%%%%%foreword.tex%%%%%%%%%%%%%%%%%%%%%%%%%%%
% sample foreword
%
% Use this file as a template for your own input.
%
%%%%%%%%%%%%%%%%%%%%%%%% Springer %%%%%%%%%%%%%%%%%%%%%%%%%%

\foreword

%% Use the template \textit{foreword.tex} together with the document class SVMono (monograph-type books) or SVMult (edited books) to style your foreword\index{foreword}. 

%% The foreword covers introductory remarks preceding the text of a book that are written by a \textit{person other than the author or editor} of the book. If applicable, the foreword precedes the preface which is written by the author or editor of the book.

This book provides highlights of the research presented at the FEniCS 2024
Conference held in Oslo, Norway in June 2024. The selected topics published
here show the typical breadth of FEniCS Project-related research as a whole,
including new software and algorithm developments, performance assessment of
the implementations on advanced computer architectures, and computational
simulation of challenging applications. This book not only confirms the
original objectives of the FEniCS Project, but it also demonstrates the
continuing vitality of the the original concepts behind it.

The FEniCS Project was started at an informal meeting at the University of
Chicago in 2003, which I attended along with a number of my graduate students.
Since then, FEniCS has grown to an international collaboration involving
developers and users at sites across the globe. Starting in 2005, the FEniCS
Project has held annual meetings at sites in Europe and the United States,
often including tutorial sessions on ``how to use it'' for beginners. During
the COVID-era, the online FEniCS conference utilized the online Gather software
that enabled and encouraged personal contact with people you had never met
before. I have had the pleasure to attend almost all of the FEniCS conferences,
with just a few exceptions.

FEniCS was one of the first software projects to address the automation of
computational mathematical modeling \cite{lrsBIBih}. This includes tools to
generate key components of scientific simulation software as well as complete
end-user systems based on these tools. Users can then write codes to solve
challenging problems in fluid dynamics, heat transfer, advanced materials, and
many other areas, including what is now known as multiphysics. Novice users at
have been able to assemble codes for complex simulations using novel models
with little help from developers. Initially, such help came primarily via
e-mail, but today online forums such as Discourse are used.

One feature of FEniCS was that it encouraged open publication of the algorithms
behind the software. This provided academic recognition that is missing from
some other software projects. This approach was copied from computer science,
where it has long been standard. FEniCS-related research was published in a
number of long-standing journals. More recently, dedicated journals devoted to
scientific software have emerged. Thus, FEniCS pioneered a culture shift
regarding how scientific software is viewed and documented.

The FEniCS concept has, since its inception, spurred various forks as people
developed different end-user interfaces and internal software implementations
to achieve their objectives. One such bifurcation involved the development of a
highly parallel version FEniCS-HPC, while the FEniCS Project targeted a broader
user interface and internal structure to attract a larger user base. A later
bifurcation led to the Firedrake Project. Most recently, a number of forked
(DOLFINx, FFCx) and new (Basix) components have been developed within the
FEniCS Project itself. Thus, the FEniCS concept has provided an environment in
which different approaches were able to flourish, and this aspect continues
today.

FEniCS, Firedrake and DUNE rely on the Unified Form Language (UFL) to provide a
language to express simulation models based on the variational form of partial
differential equations (PDEs). This possibility was already recognized in the
first edition of the book ~\cite{lrsBIBcq} at the end of the first paragraph in
Chapter 0. This observation was, a decade later, demonstrated in the system
Analysa~\cite{lrsBIBfo}. Analysa~\cite{lrsBIBfo} provided only a limited family
of finite elements, including arbitrary-order Lagrange elements, automatically
by a specific algorithm. One feature of Analysa that is not yet replicated in
FEniCS (or Firedrake) is an algebra of domains and corresponding finite element
spaces. It was possible in Analysa to define a space of functions on the
boundary, on the interior, and to define linear functionals and matrices
related to these spaces. This provided a way to implement the linear algebra
for problem solution precisely. Perhaps such functionality can appear in future
automated PDE software.

A different approach to a language for PDEs appeared earlier in the finite
difference language FIDIL~\cite{hilfinger1989fidil}. NGSolve has also developed
its own syntax, similar to UFL, with solvers relatively straightforward to port
from one code to another.

One valuable aspect of FEniCS and related solvers is that they allow algorithm
developers a way to make their advances available to a wide audience. A code
written from scratch is hard to use and adapt by novices, and often goes
underutilized. This approach has become quite influential among finite element
codes.

FEniCS has allowed the development and testing of new ideas by a very large
community, including people with minimal software engineering training. Often a
new technique can be tested numerically before expending the effort to
understand the new method analytically. For example, the Robin method described
in \cite{lrsBIBig} was first tested on a simple problem before attempting to
show that the method was well posed. In fact, the proof of that took
significant time, since it required discovering a new approach to analyzing
finite element methods. Without the assurance that the method actually worked
in practice, we likely would have abandoned the search for a rigorous
explanation of its behavior.

Today users have many options to choose from when approaching the
implementation of a finite element model. This includes Firedrake, NGSolve, and
the modern DOLFINx-based version of the FEniCS Project. The legacy version of
FEniCS (DOLFIN) from 2019 is also still widely used. This is not the forum to
compare and contrast these different choices, but one can find some guidance
from online discussions.

The current book gives a glimpse of the leading edge of FEniCS-related
research. It is a good way to find out current research topics and get a sense
of directions for the FEniCS Project for the future.

\vspace{\baselineskip}
\begin{flushright}\noindent
Chicago, USA, 22nd October 2025\hfill {\it L. Ridgway Scott}\\
\end{flushright}

\bibliographystyle{spbasic}
\bibliography{foreword.bib}
