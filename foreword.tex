%%%%%%%%%%%%%%%%%%%%%%foreword.tex%%%%%%%%%%%%%%%%%%%%%%%%%%%
% sample foreword
%
% Use this file as a template for your own input.
%
%%%%%%%%%%%%%%%%%%%%%%%% Springer %%%%%%%%%%%%%%%%%%%%%%%%%%

\foreword

%% Use the template \textit{foreword.tex} together with the document class SVMono (monograph-type books) or SVMult (edited books) to style your foreword\index{foreword}. 

%% The foreword covers introductory remarks preceding the text of a book that are written by a \textit{person other than the author or editor} of the book. If applicable, the foreword precedes the preface which is written by the author or editor of the book.

This book provides highlights of the research presented at
the FEniCS 24 meeting held in Oslo, Norway in June, 2024.
The selected topics published here show the typical breadth of FEniCS research as a whole,
including
\begin{itemize}
\item
new software and algorithm developments, including some performance assessment
of the implementations
on advanced computer architectures, and
\item
computational simulation of challenging applications using some version of FEniCS (dolfin).
\end{itemize}
This book not only confirms the original objectives of the FEniCS Project,
but it also demonstrates the continuing vitality of the the original concepts
behind the FEniCS Project.

The FEniCS Project was started at an informal meeting at the
University of Chicago in 2003. It has grown to an international
collaboration involving primary developers and users at sites in England, 
Europe, Asia, and the Americas. 
Discounting the Covid era, the FEniCS Project has held an annual, 
in-person meeting
at sites in England, Europe, and the United States, often including tutorial
sessions on ``how to use it'' for beginners.
The online FEniCS meeting during the Covid era utilized some special software
that enabled and encouraged personal contact with people you had never met before.
I have had the pleasure to attend almost all of the FEniCS meetings, with
just a few exceptions.
This book is the first refereed conference proceedings for the FEniCS meetings.

FEniCS was one of the first projects to address the automation 
of computational mathematical modeling \cite{lrsBIBih}.
This includes tools to generate key components of 
scientific simulation software as well as complete end-user 
systems based on these tools. The end-user codes are being 
used to solve challenging problems in fluid dynamics, heat 
transfer, advanced materials, and many other areas, including 
what is now known as multiphysics.
Novice users at remote sites 
have been able to assemble codes for complex simulations 
using novel models with little help from developers.
Initially, such help came primarily via e-mail, but today
online systems such as 
{\tt https://fenicsproject.discourse.group}
are used.

One feature of FEniCS was that it encouraged publication of the
algorithms behind the software.
This provided academic recognition that is missing from some
other software projects.
This approach was copied from computer science, where it is standard.
There are long-standing journals in which early FEniCS research
was published.
More recently, journals devoted to scientific software have emerged.
Thus, FEniCS pioneered a culture shift regarding how software is 
viewed and documented.

FEniCS has, since its inception, involved various bifurcations
as people developed different end-user interfaces and internal
software implementations to achieve different targeted objectives.
One such bifurcation involved the development of a high-performance
(highly parallel) version, while the main branch targeted a broader
user interface and internal structure to attract a larger user base.
A later bifurcation led to the Firedrake project.
Thus, the FEniCS Project has provided an environment in which different
approaches were able to flourish, and this feature continues today,
despite the fact that legacy FEniCS software is still widely used.

FEniCS (and Firedrake and others) rely on the variational formulation of 
partial differential equations (PDEs) to provide a language to express
simulation models.
This possibility was recognized already in the first edition of the book
~\cite{lrsBIBcq} (see the end of the first paragraph in Chapter 0).
This observation was, a decade later, demonstrated in the system
Analysa~\cite{lrsBIBfo}.
A different approach to a language for PDEs appeared earlier in the
finite difference language FIDIL~\cite{hilfinger1989fidil}.

NGSolve uses syntax similar to that of FEniCS and Firedrake, so that
codes written for one system are easy to port to another.
One valuable aspect of FEniCS and other projects is that they allow
algorithm developers a way to make their advances available to
a wide audience.
A code written from scratch is hard to use by novices, and such
codes often are underutilized.

Analysa~\cite{lrsBIBfo} provided only a limited family of elements,
but arbitrary order Lagrange elements were available, generated
automatically by a specific algorithm.
One feature of Analysa that is not yet replicated in FEniCS (or Firedrake)
is an algebra of domains and corresponding finite element spaces.
It was possible in Analysa to define a space of functions on the boundary,
on the interior, and to define linear functionals and matrices
related to these spaces.
This provided a way to implement the linear algebra for problem
solution precisely.
Perhaps such functionality can appear in future automated PDE software.


FEniCS has allowed the development and testing of new ideas by a
very large community, including people with minimal software training.
Often a new technique can be tested numerically before expending
the effort to understand the new method analytically.
For example, the Robin method described in \cite{lrsBIBig} was first
tested on a simple problem before attempting to show that the method
was well posed.
In fact, the proof of that took significant time, since it required
discovering a new approach to analyzing finite element methods.
Without the assurance that the method actually worked in practice,
we likely would have abandoned the search for a rigorous explanation
of its behavior.

Today users have many options to choose from when approaching the 
implementation of a technical model.
This includes Firedrake, NGSolve, and various flavors of FEniCS.
The legacy version of FEniCS (dolfin) from 2019 is still widely used.
This is not the forum to compare and contrast these different choices,
but one can find some guidance from online discussions.

The current volume gives a glimpse of the leading edge
of FEniCS-related research.
It is a good way to find out current research topics and get a sense
of directions for the FEniCS Project for the future.

Ridgway Scott was one
of the founders of FEniCS, and Matt Knepley (of the Computation
Institute) has become one of the major developers. Two
graduate students in CS are also primary developers. Current
efforts are underway to extend FEniCS from traditional two- and 
three-dimensional models in mechanics to arbitrary dimensional 
systems as arise in quantum mechanics. FEniCS has also hosted
conferences that feature automated scientific software
development outside its primary domain of solving partial
differential equations.




\vspace{\baselineskip}
\begin{flushright}\noindent
Chicago, USA, 22nd October 2025\hfill {\it L. Ridgway Scott}\\
\end{flushright}



\bibliography{histor}
%\bibliographystyle{IEEEtran}
\bibliographystyle{plainnat}
%\bibliographystyle{unsrtnat}
