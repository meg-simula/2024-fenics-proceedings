\graphicspath{{chapters/munthe/graphics/}}


\title{Growth and Remodelling Package in FEniCSx}

\author{Karl Munthe, Henrik N.T. Finsberg, Samuel T. Wall, and Joakim Sundnes}

\institute{K. Munthe \email{karlfredrik@simula.no} \at Simula Research Laboratory, Oslo, Norway and \\ Department of Informatics, University of Oslo, Norway 
\and
H. Finsberg \at Simula Research Laboratory, Oslo, Norway
\and
S. Wall \at Simula Research Laboratory, Oslo, Norway
\and
J. Sundnes \at Simula Research Laboratory, Oslo, Norway, and \\ Department of Informatics, University of Oslo, Norway}

\maketitle
\abstract{The heart is a dynamic organ that changes its size and shape to regulate its behaviour to the demands of the body, which can change, for example, through body growth, exercise, or the onset of disease. Different models have been proposed to capture various types of cardiac growth resulting from mechanical stimuli, but the models have rarely been compared systematically. In this manuscript, we present a framework implemented in FEniCSx that allows one to quickly run simulations of growth and remodelling with different material models and different growth laws. We present and compare the growth predicted by each model for a set of simple experiments and compare the results to the literature. All the code can be found at \url{https://github.com/karlfm/Growth-and-Remodeling-in-FEniCSx}.}
\vspace{12pt}
\section{Introduction}
In classical continuum mechanics, one normally studies the mechanics of bodies where mass, linear momentum, angular momentum, and energy are conserved. This approach has been extremely successful and is the bedrock for traditional engineering disciplines, but it does not accurately capture aspects of how living organisms change with respect to their environment. One of the unique features of biological material is its ability to grow and evolve by adding or removing mass. Understanding how biological matter grows and what drives the growth is important, not only to understand normal growth and development but also when and how growth can become non-compensatory and drive disease.\par It has been known for a long time that the growth of organs, such as the heart, is regulated at least partly by the forces applied to it \citep{Hsu1968}. This understanding has led to the formulation of growth laws that link growth and remodelling to local stress or strain. \par
In this chapter, we introduce a package written in FEniCSx that allows one to easily model the growth and remodelling of biological tissue, with the aim of quickly testing combinations of growth models and material models. Allowing researchers to systematically test combinations of material models and growth tensors will aid in the discovery of more accurate models of growth and remodelling phenomena of biological tissue.
\section{Methods}
\subsection{Growth and Remodelling in Continuum Mechanics}
Consider a solid body that is continuously and smoothly deforming from one configuration to another and denote the initial configuration (also called the reference configuration) as $\mathcal{M}$ and the current configuration as $\mathcal{N}$. We denote a point in $\mathcal{M}$ with uppercase letters $\mathbf{X} = (X, Y, Z)$ and a point in $\mathcal{N}$ with lowercase letters $\mathbf{x} = (x, y, z)$\footnote{Apart from the letters $X$, $Y$, and $Z$, all uppercase letters represent tensors.}. A point in $\mathcal{M}$ can be mapped to a point in $\mathcal{N}$ by a motion $\phi(\mathbf{X}): \mathbf{X} \rightarrow \mathbf{x}$, which is a diffeomorphism. We can map a vector from the reference configuration to the current configuration via the pushforward of $\phi$, which is commonly denoted by $\mathbf{F}$ and is referred to as the deformation gradient, computed as $\mathbf{F} = \partial\mathbf{x}/\partial\mathbf{X}$. The displacement field $\mathbf{u} = \mathbf{x} - \mathbf{X}$ is a vector field describing the displacement of each point $\mathbf{X}$ in the reference configuration to its location $\mathbf{x}$ in the deformed configuration. 
Most mechanical models of the heart assume that the tissue is hyperelastic, meaning that deformations of the material conserve their energy, and when any load on the tissue is removed, the material returns to its original reference shape. Growth, on the other hand, represents a permanent change of the unloaded reference configuration and cannot be modelled as an elastic deformation. Instead, it is commonly modelled using the framework of plastic or elastoplastic deformations. The most common approach, introduced by \cite{Rodriguez1994}, is to multiplicatively split the deformation tensor into an elastic part and an inelastic growth part, 
\begin{equation}
\label{eq: multiplicative split}
    \mathbf{F} = \mathbf{F}_e\mathbf{F}_g,
\end{equation}
where $\mathbf{F}_e$ is the elastic deformation and $\mathbf{F}_g$ is the plastic deformation that represents growth. 

One interpretation of \ref{eq: multiplicative split} is that the material first deforms by $\mathbf{F}_g$ in a way that does not cause stress but might cause incompatibilities in the form of discontinuities or overlapping material. The deformation described by $\mathbf{F}_g$ leads to an unphysical intermediate configuration, which is then deformed by $\mathbf{F}_e$ in a way that removes the unphysical characteristics that resulted from $\mathbf{F}_g$ but adds residual stress. Sufficient conditions for the existence of intermediate configurations such as $\mathbf{F}_g$ are discussed in \cite{Goodbrake2021}. We assume that the deformation described by $\mathbf{F}_e$ is hyperelastic such that we can obtain the first Piola–-Kirchhoff stress tensor by differentiating a strain energy function,
\begin{equation}
\label{eq: stress}
    \mathbf{P} = \frac{\partial\Psi}{\partial \mathbf{F}_e}.
\end{equation}

We typically describe the growth in terms of multiple growth steps, given by
\begin{equation*}
    \mathbf{F}_g^{i + 1} = \mathbf{F}_g^i\mathbf{F}_g^\mathrm{inc},
\end{equation*}
where $\mathbf{F}_g^\mathrm{inc}$ is the incremental growth tensor describing the growth occurring in one step and $\mathbf{F}_g^i$ is the cumulative growth after $i$ steps. The initial growth tensor, $\mathbf{F}_g^0$, is set to the identity tensor. The cumulative growth deformation tensor after $n$ steps is given by
\begin{equation*}
    \mathbf{F}_g^{n} = \mathbf{F}_{g}^\mathrm{inc}\vert_{t=0}\mathbf{F}_{g}^\mathrm{inc}\vert_{t=1} \cdots \mathbf{F}_{g}^\mathrm{inc}\vert_{t=n},
\end{equation*}  
where $\mathbf{F}_{g}^\mathrm{inc}\vert_{t=i}$ means $\mathbf{F}_{g}^\mathrm{inc}$ at the $i$th step. Note that the $i$th incremental growth tensor is dependent on the stress or strain that occurred in the ($i-1$)th growth step \citep{Goriely2007}. It is common to assume that the growth tensor is diagonal and to express the incremental growth tensor in terms of fibre, cross-fibre, and normal directions: 
\begin{equation*}
    \mathbf{F}_g^\mathrm{inc} = F^\mathrm{inc}_{g,f}\mathbf{e}_f\otimes \mathbf{e}_f + F^\mathrm{inc}_{g,c}\mathbf{e}_c\otimes \mathbf{e}_c + F^\mathrm{inc}_{g,n}\mathbf{e}_n\otimes \mathbf{e}_n,
\end{equation*}
where the $F^\mathrm{inc}_{g, i}$ terms for  $i = \{f, c, n\}$ are functions of either stress or strain, and the $\mathbf{e}_i$ terms for $i = {\{f, c, n\}}$ are orthonormal basis vectors in the fibre, cross-fibre, and normal directions, respectively. 

The incremental growth tensor depends on the local stress or strain, which is determined by solving for mechanical equilibrium at each growth step: 
\begin{equation} \label{eq: system of equations}
\begin{aligned}
    \mathbf{F} & = \mathbf{F}_e\mathbf{F}_g && \text{in } \mathcal{M} ,\\
    \nabla\cdot\mathbf{P} & = 0 && \text{in } \mathcal{M}, \\
    \mathbf{P}\cdot \nu & = 0 && \text{on } \partial\mathcal{M}_N, \\
    \mathbf{u} & = g_D && \text{on } \partial\mathcal{M}_D,
\end{aligned}
\end{equation} 
where $\nu$ is a surface normal vector, and $\partial\mathcal{M}_N$ and $\partial\mathcal{M}_D$ denote the boundaries that are prescribed Neumann and Dirichlet boundary conditions, respectively. 

The growth laws that determine $F_g$ and the strain energy $\Psi$ in 
(\ref{eq: stress}) still need to be specified. We will define the growth laws in the next section, but we use a nearly incompressible neo-Hookean model for all the 
experiments, so (\ref{eq: stress}) becomes
\begin{align*}
    \mathbf{P} &= \frac{\partial\Psi_\text{iso}}{\partial \mathbf{F}_e} + \frac{\partial\Psi_\text{vol}}{\partial \mathbf{F}_e}, \\
    \mathbf{P} &= \frac{\partial}{\partial \mathbf{F}_e}\left[\frac{\mu}{2}\left(\mathrm{tr}\mathbf{\bar{C}} - 3\right) + \kappa(J-1)^2\right],
\end{align*}
where $\mu$ and $\kappa$ are material parameters, and $\Psi_\text{iso}$ and $\Psi_\text{vol}$ are the isochoric (distortional) and volumetric (dilational) parts of the strain energy function, respectively. To decouple the energy stored in the body as a result of volume-preserving deformation and non--volume-preserving deformation, we introduce $\mathbf{\bar{F}}_e = \mathbf{F}_eJ^{-1/3}$, whose determinant is equal to one. The isochoric right Cauchy--Green deformation tensor, $\mathbf{\bar{C}}$, is calculated as $\mathbf{\bar{C}} = \mathbf{\bar{F}}_e^\top \mathbf{\bar{F}}_e$. Now, $\partial\Psi_\text{iso}/\partial \mathbf{F}_e = 0$ only if the deformation preserves the shape, and $\partial\Psi_\text{vol}/\partial \mathbf{F}_e = 0$ only if the deformation preserves the volume (for more details, see Chapter 6 of \cite{Holzapfel2002}). \par 
For further information about continuum mechanics, we recommend \cite{Marsden1983} and \cite{Holzapfel2002}, and for further information about growth and remodelling, we recommend \cite{Goriely2017} and \cite{Yavari2010}.

\subsection{Numerical Implementation}
 
Algorithm \ref{alg:growth_deform} gives an overview of the steps involved in the solution
of the growth model equations. For stress-based growth, one would update the stress tensor rather than the strain tensor, and for additive growth laws, one would sum the cumulative and incremental growth tensors instead of multiplying them. \par
\begin{algorithm} 
    \caption{The growth tensor and stress/strain tensor are updated at each growth step. Both $\mathbf{F}_\mathrm{e}$ and $\mathbf{F}_g^\mathrm{inc}$ are dependent on $\mathbf{u}$.}\label{alg:growth_deform}
    \SetAlgoLined
    \For{each time step}{
        Solve (\ref{eq: system of equations}) for the displacement $\mathbf{u}$\;
        Update the stress/strain tensor using the obtained displacement $\mathbf{u}$.\;
        Update the growth tensor using the stress/strain tensor from the previous line.\;
    }
\end{algorithm} 
\emph{Constructing the weak form}: We multiply $\nabla\cdot\mathbf{P}$ by a test function, which we set to be in the same function space as $\mathbf{u}$, and integrate over a discretization of $\mathcal{M}$. By applying integration by parts, we obtain
\begin{align}
    \label{eq: weak conservation of momentum}
    \int_\Omega(\nabla\cdot\mathbf{P})\cdot\eta d\mathbf{X} &= 0  \notag\\
    \int_\Omega \mathbf{P} : \nabla\eta d\mathbf{X} &= \int_{\partial\Omega}\mathbf{P}\cdot\eta \cdot \nu dA.
\end{align}
Since we are using test functions $\eta$ that vanish on $\partial_D\mathcal{M}$ and the normal component of $\mathbf{P}$ is zero on $\partial_N\mathcal{M}$, we can set the boundary integral to zero. Then (\ref{eq: weak conservation of momentum}) is solved using FEniCSx \citep{DOLFINx}. 
\par
\emph{Iteratively solving the conservation of momentum}: We now solve (\ref{eq: weak conservation of momentum}) for the displacement $\mathbf{u}$, which we can use to compute all the necessary variables. We use tetrahedral, second-order, continuous Lagrange elements to approximate $\mathbf{u}$, and first-order, discontinuous Lagrange elements to approximate $\mathbf{F}_e$ and $\mathbf{F}_g$. This is a common numerical scheme in cardiac mechanics that has been demonstrated to avoid locking \citep{oliveira2016comparison}. 

\subsection{Solving Growth Laws on the Unit Cube}
\label{subsec: simulations}
In the simulations we have run, we have aligned the $x$-axis with the fibre direction, with the $y$- and $z$-axes as the cross-fibre and normal directions, respectively. For consistency with the literature, we use $\mathbf{e}_f$, $\mathbf{e}_c$, and $\mathbf{e}_n$ to denote the unit vectors in the $(x, y, z)$ directions, respectively.  \par
\emph{Boundary conditions:} We set the following boundary conditions:
\begin{align*}
    g_D = \begin{cases}
        u &= \begin{cases}
            0 & \text{on } x = 0, \\
            u_D & \text{on } x = 1,
        \end{cases} \\
        v &= 0 \qquad \ \ \text{on } y = 0, \\
        w &= 0 \qquad \ \ \text{on } z = 0,
    \end{cases}
\end{align*}
where $u$, $v$, and $w$ are the displacement in the $x$, $y$, and $z$ directions, respectively, and $u_D$ specifies by how much the body is displaced. 
\par

\emph{Numerical simulations:} We ran two simulations, one with a 10\% stretch and one with a 10\% compression, corresponding to $u_D = 0.1$ and $u_D = -0.1$, respectively. For the GCG model, $F_{g,c,\mathrm{max}}$ was set to 1.2 in the stretch simulation and 0.8 in the compression simulation (see Table \ref{tab:growth models}). We set $\mu = 15$ kPa and $\kappa = 100$ kPa.
\subsection{Growth Models}
\label{sub:different models} 
In this paper, we compare five growth models that we have taken from \cite{Taber1998}, \cite{Kroon2009}, \cite{Goktepe}, and \cite{Kerckhoffs2012}. The growth models are presented in Table \ref{tab:growth models}, where the LT2 model is from \cite{Taber1998}, the KFR model is from \cite{Kroon2009}, the GEG and GCG models are from \cite{Goktepe}, and the KOM model is from \cite{Kerckhoffs2012}. Each growth model has a set point that determined the homeostatic level of either stress, stretch, or strain. When the stress, stretch, or strain reaches the set point, growth will cease to occur. If this does not happen, the body will grow indefinitely, which we call runaway growth. We used the same variables as in the original works, except for the GCG model, where we scale the variables to more accurately fit the shear modulus used here. The values are tabulated in Table \ref{tab:parameters}.
%\newgeometry{right=1cm, left=1cm}%,right=1cm,top=1cm,bottom=1cm}
\begin{table}
\centering
\makebox[\textwidth][c]{
\renewcommand{\arraystretch}{4}
\begin{tabular}{|c||c|c|c|}
\hline \hline
 & $F_{g,f}^{i+1}$ & $F_{g,n}^{i+1}$ & $F_{g,c}^{i+1}$ \\
\hline \hline
LT2 & $\displaystyle F_{g,f}^i\left(\frac{\sigma_{\theta p} - \sigma_{p,0}}{T\sigma_{p,0}} + 1\right)$ & $\displaystyle F_{g,n}^i\left(\frac{\sigma_{\theta a} - \sigma_{a,0}}{T\sigma_{a,0}} + 1\right)$ & $\displaystyle 1$ \\
\hline
KFR & $\displaystyle F_{g,f}^i(\beta(\sqrt{2 E_{ff} + 1} - 1 - s_\mathrm{hom}) + 1)^{1/3}$ & $\displaystyle F_{g,n}^i(\beta(\sqrt{2 E_{ff} + 1} - 1 - s_\mathrm{hom}) + 1)^{1/3}$ & $\displaystyle F_{g,c}^i(\beta(\sqrt{2 E_{ff} + 1} - 1 - s_\mathrm{hom}) + 1)^{1/3}$ \\
\hline
GEG & $\displaystyle \frac{1}{\tau}\left(\frac{F_{g,f,\mathrm{max}} - F_{g,f}^i}{F_{g,f,\mathrm{max}} - 1}\right)^\gamma(F_{e, f}^i - \lambda^\text{crit}) + F^i_{g, f}$ & $\displaystyle 1$ & $\displaystyle 1$ \\
\hline
GCG & $\displaystyle 1$ & $\displaystyle  \frac{1}{\tau}\left(\frac{F_{g,c,\mathrm{max}} - F_{g,c}^i}{F_{g,c,\mathrm{max}} - 1}\right)^\gamma(\tr(\mathbf{M}) - p^\mathrm{crit}) + F^i_{g, c}$ & $\displaystyle 1$ \\
\hline
KOM & $\displaystyle \begin{cases}
        F_{g,f}^{i}k_{ff}\frac{f_\mathrm{ff, max}\Delta t_\text{growth}}{1 + \exp(-f_f(s_\mathrm{l}-s_{l,50}))} + 1, \qquad s_\mathrm{l} \geq 0\\
        F_{g,f}^{i}\frac{-f_\mathrm{ff, max}\Delta t_\text{growth}}{1 + \exp(f_f(s_\mathrm{l}+s_{l,50}))} + 1, \qquad s_\mathrm{l} < 0
    \end{cases} $ & $\displaystyle \begin{cases}
        F_{g,c}^{i}\sqrt{k_{cc}\frac{f_{cc,\mathrm{max}}\Delta t_\text{growth}}{1 + \exp(-c_\mathrm{f}(s_\mathrm{t}-s_{t,50}))} + 1}, \qquad s_\mathrm{t} \geq 0\\
        F_{g,c}^{i}\sqrt{\frac{-f_{cc,\mathrm{max}}\Delta t_\text{growth}}{1 + \exp(c_\mathrm{f}(s_\mathrm{t}+s_{t,50}))} + 1}, \qquad s_\mathrm{t} < 0 
    \end{cases} $ & $\displaystyle \begin{cases}
        F_{g,c}^{i}\sqrt{k_{cc}\frac{f_{cc,\mathrm{max}}\Delta t_\text{growth}}{1 + \exp(-c_\mathrm{f}(s_\mathrm{t}-s_{t,50}))} + 1}, \qquad s_\mathrm{t} \geq 0\\
        F_{g,c}^{i}\sqrt{\frac{-f_{cc,\mathrm{max}}\Delta t_\text{growth}}{1 + \exp(c_\mathrm{f}(s_\mathrm{t}+s_{t,50}))} + 1}, \qquad s_\mathrm{t} < 0 
    \end{cases} $ \\
\hline
\end{tabular}
}
\caption{The terms $F_{g,f}$, $F_{g,c}$, and $F_{g,n}$  for each of the five models. The parameters $T$, $\beta$, $\tau$, and $\Delta t$ simply determine the rate of growth and can be tuned to match the growth rate of the data obtained from experiments.}
\label{tab:growth models}
\end{table}
%\restoregeometry
\begin{table}[htbp]
    \centering
    \begin{tabular}{|l|l|}
    \hline
    \textbf{Model} & \textbf{Parameters} \\
    \hline
    \textbf{LT2} &   $\sigma_{a,0} = 30$ [kPa], $\sigma_{p,0} = 3$ [kPa], $T = 10^{-4}$ \\ \hline
    \textbf{KFR} &  $s_\mathrm{hom} = 0.13, \beta = 10^{-2}$ \\ \hline
    \textbf{GEG} &  $F_{g,f,\mathrm{max}}=1.5$, $\lambda^\mathrm{crit}=1.01$, $\gamma = 2$, $\tau = 10^2$ \\ \hline
    \textbf{GCG} &  $F_{g,c,\mathrm{max}}=1.2$ and $0.8$,  $p^\mathrm{crit}=0.12, \gamma = 2, \tau = 10^4$ \\ \hline
    \textbf{KOM} &  $f_\mathrm{ff,max} =0.31$ [1/days], $f_f = 150$, $s_{l50} = 0.06$, $F_{ff50} = 1.35$, $f_{l,\mathrm{slope}} = 40$, $f_\mathrm{ff,max} = 0.1$ [1/days] \\
        & $c_\mathrm{f} = 75$, $s_\mathrm{t50} = 0.07$, $F_\mathrm{cc50} = 1.28$, $c_\mathrm{th,slope} = 60$, $E_{ff,\mathrm{set}} = 0$, 
        $E_\mathrm{cross,\mathrm{set}} = 0$, $\Delta t = 10^{-2}$ [days] \\ \hline
    \end{tabular}
    \caption{Model parameters for the growth models. The terms $T$, $\beta$, $\tau$, and $\Delta t$ determine the speed of growth.}
    \label{tab:parameters}
\end{table}
In the LT2 model, $\sigma_{p,0}$ and $\sigma_{a,0}$ are set points for the passive and active fibre stresses at equilibrium, and $\sigma_{\theta p}$ and $\sigma_{\theta a}$ are the active and passive fibre stresses, respectively. In the simulations we have run, we have only used the passive component of $\sigma$ and have set $\sigma_a = 0$. For the KFR model, $s_\mathrm{hom}$ is the strain set point. For the GEG and GCG models, $F_{g,f,\mathrm{max}}$ and $F_{g,c,\mathrm{max}}$ are the maximum amounts of growth allowed to occur, respectively. The term $\mathbf{M}$ is the Mandel stress, which is defined as 
\begin{equation*}
    \mathbf{M} = \mathbf{F}^\top \mathbf{P},
\end{equation*}
and $p^\mathrm{crit}$ is the stress set point. The term $\lambda^\text{crit}$ is the strain set point. For the KOM model, $k_{ff}$ and $k_{cc}$ are defined as
\begin{align*}
    k_{ff} &= \frac{1}{1 + \exp(f_\text{length,slope}(\mathbf{F}_{g,ff}^i - F_{ff,50}))}, \\
    k_{cc} &= \frac{1}{1 + \exp(c_\text{thickness,slope}(\mathbf{F}_{g,cc}^i - F_{cc,50}))},
\end{align*}
and $s_\mathrm{l}$ and $s_\mathrm{t}$ are defined as
\begin{align*}
    s_\mathrm{l} &= \max(E_{ff}) - E_{ff, \mathrm{set}}, \\
    s_\mathrm{t} &= \min(E_\text{cross, max}) - E_\mathrm{cross, set},
\end{align*}
where $E_{ij}$ is the Lagrange strain tensor, $E_{ff}$ is the strain in the fibre direction, and $E_\text{cross, max}$ is the maximum algebraic maximum principal strain of the matrix \citep{Witzenburg2018}:
\begin{equation*}
    E_\text{cross} = \begin{pmatrix}
        E_{cc} & E_{cr} \\
        E_{rc} & E_{rr}
    \end{pmatrix},
\end{equation*}
and $E_{ff, \mathrm{set}}$ and $E_\mathrm{cross, set}$ are set points. \par
Growth stops for the LT2 model when $\sigma_{\theta} = \sigma_{0}$. For the KOM model, since $k_{cc}$ and $k_{ff}$ are logistic functions, growth is bounded from above and below, inhibiting runaway growth. Finally, for the KFR model, it does not appear obvious that it will not obtain runaway growth, and other simulations setups that were tested did result in runaway growth, even though the one we present here does not.

\section{Results}
The data we collected from the simulations described in Section \ref{subsec: simulations} were the stretch and growth that occurred in the middle of the cube. The results are depicted in Figs. \ref{fig:10p_stretch} and \ref{fig:10p_compression}. The top row of each figure displays the fibre and cross-fibre components of the growth tensor, $\mathbf{F}_g$, and the bottom row displays the fibre and cross-fibre components of the elastic deformation tensor, $\mathbf{F}_e$. In the simulations we ran, $\mathbf{F}_e$ is diagonal, so the components of $\mathbf{F}_e$ are the principal stretches. This is because $\sqrt{\mathbf{e}_i^\top\mathbf{C}_e\mathbf{e}_i}$ is the principal stretch in the $i$th direction, and $\sqrt{\mathbf{e}_i^\top\mathbf{C}_e\mathbf{e}_i} = \sqrt{\mathbf{e}_i^\top\mathbf{F}_e^\top\mathbf{F}_e\mathbf{e}_i} = \mathbf{F}_e\mathbf{e}_i$. By the same reasoning, the diagonal components of $\mathbf{F}_g$ (which are the only nonzero components), denote the growth in the fibre, cross-fibre, and normal directions. Increasing $\kappa$ did not yield qualitatively different results. \par
The GCG model seems to be converging to $F_{g,c,\mathrm{max}}$, and the GEG model seems to have converged because it reached $\lambda^\text{crit}$. The reason the GCG model is showing growth oppositely compared to the GEG model is probably because the GEG model was created to model growth triggered by volume overload, whereas the GCG model was created to capture growth triggered by pressure overload. The KFR model showed equal amounts of growth in each direction and stabilized. It is not clear under what conditions the KFR model should be stable, because $s_\mathrm{hom}$ is the same in each direction. When we ran simulations with other boundary conditions, the solution diverged. The KOM model was stable for many different types of boundary conditions, but it is the most computationally expensive to run.\par 
\begin{figure}[h]
    \centering
    \includegraphics[width=\textwidth]{10p_stretch_3.png}
    \caption{Growth and stretch predicted by 10\% stretch in the fibre direction}
    \label{fig:10p_stretch}
\end{figure}
\begin{figure}[h]
    \centering
    \includegraphics[width=\textwidth]{10p_compression_3.png}
    \caption{Growth and stretch predicted by 10\% compression in the fibre direction}
    \label{fig:10p_compression}
\end{figure}    

\section{Conclusion and Future Work}
We have implemented a general growth and remodelling framework using the FEniCSx program in Python. The goal is to easily change material models and growth models. This will allow researchers to compare their models with other models in the field. Future work will include the implementation of more complex geometries, more growth laws, and more material models. The models we have used in this chapter are not derived from the dissipation equation but are, instead, phenomenologically derived growth laws, and future work should investigate whether they satisfy the laws of thermodynamics. Another avenue of future research we wish to pursue is examining constrained mixture models, which model how changes in various constituents influence the characteristics of tissue. We also wish to add models that have more mathematically sophisticated stopping criteria, such as those developed by \cite{Erlich2023}. The authors use an energy penalty to construct a stopping criterion, and \cite{Erlich2024} look into how curvature\footnote{The intrinsic three-dimensional curvature, not the two-dimensional curvature of the surface of the body.} in the reference configuration could be used as a stopping criterion. Future work will also implement the growth models on geometries with fibres. We tried running the models on various fibre orientations and found them to be extremely sensitive to fibres that varied throughout the domain. Preliminary results indicate that some of the models do not converge to a steady state for relatively small perturbations of the variables or if the fibres are not well aligned with the body, something we plan on quantifying in the future. \par

When this package is further developed, we aim to add it to the Pulse package. \footnote{See https://github.com/finsberg/fenicsx-pulse.} \par
The models we compared were developed to capture different aspects of growth and were tuned to be used on different material models. An apples-to-apples comparison might therefore be unfair. Furthermore, the growth models we used do not take into account residual stresses that exist within the material before or after growth. \par
Finally, experimental data are needed to verify which models are accurate or to capture the correct phenomena of growing cardiac tissue.
% \newpage
\bibliographystyle{spbasic}
\bibliography{chapters/munthe/bibliography.bib}
% \printbibliography
% \end{document}



