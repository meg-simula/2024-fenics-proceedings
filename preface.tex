%%%%%%%%%%%%%%%%%%%%%%preface.tex%%%%%%%%%%%%%%%%%%%%%%%%%%%%%%%%%%%%%%%%%
% sample preface
%
% Use this file as a template for your own input.
%
%%%%%%%%%%%%%%%%%%%%%%%% Springer %%%%%%%%%%%%%%%%%%%%%%%%%%

\preface

The FEniCS Conference 2024 took place between the 12th and 14th June 2024 at
the Simula Research Laboratory in Oslo, Norway. It was the 20th---or possibly
---the 19th FEniCS Conference, with official records of the 2007 edition remaining
elusive, despite some senior community members claiming that they attended. To
avoid future confusion, the 2024 edition of the conference will be permanently
marked by this first peer-reviewed conference proceedings, published as part of
the \emph{Simula SpringerBriefs on Computing} series.

The conference brought together around eighty developers, existing and
potential users of the FEniCS ecosystem, as well as mathematicians, computer
scientists and application domain specialists interested in numerical methods,
their implementation and applications. The talks were held in a single track in
the beautifully appointed Hans Petter Langtangen room in the Simula building in
central Oslo. It was a fitting venue, honouring a much missed colleague whose
contributions to the scientific computing community remain deeply appreciated.

All contributors to the conference were invited to submit a chapter for peer
review. From among the forty-nine talks and ten posters presented, twelve
contributors expressed their intent to submit a chapter, and nine were accepted
following single-blind peer review. These chapters reflect both the diversity
and the ongoing vitality of the FEniCS community, highlighting ongoing
developments, the broad usage across all areas of scientific study, and the
open collaborative spirit that has characterised the project since its
inception at the University of Chicago in 2003.

The conference featured the traditional FEniCS competitions, recognising
excellence in the categories of:
\begin{itemize}
	\item \emph{Best poster}. Alena Jarolímová, Charles University, Czech
		Republic for the poster ``Determination of Navier’s slip
		parameter and the inflow velocity using variational data
		assimilation''.
	\item \emph{Best presentation by a doctoral student}. Alexandre Guibert,
		University of California San Diego, USA, with the presentation
		``Strongly coupled electrochemical-thermal-fluid models of a
		battery pack using FEniCS''.
	\item \emph{Best presentation by a postdoctoral researcher}. Igor
		Tominec, Stockholm University, Sweden, with the presentation ``On the
		Stokes problem well-posedness under pressure Dirichlet boundary
		conditions''.
	\item \emph{Nate Sime's award for exceptional FEniCS visualization}.
		Marc Hirschvogel, Politecnico di Milano, Italy, contained in the
		presentation ``Block Preconditioners for (Moving Domain) Fluid
		Dynamics Coupled to Physics- and Projection-based Reduced
		Models''.
\end{itemize}
Thanks to the generous support of the Ridgway Scott Foundation and Nate Sime,
each winner received a cash prize in recognition of their exceptional efforts.

Three travel awards were supported by NumFOCUS, Inc., which has served as
fiscal sponsor of the FEniCS Project since 2017, and the poster session by Flax
\& Teal Ltd.\ .

We would like to thank all authors, reviewers and sponsors for their
contributions and engagement, and to the organisers and local hosts for
creating such a welcoming environment. Their combined efforts exemplify the
openness that continues to drive the FEniCS Project forward.

\vspace{\baselineskip}
\begin{flushright}\noindent
Oslo, Norway, 22nd October 2025
\hfill{\it Jørgen S.\ Dokken, Henrik N.\ T.\ Finsberg, Jack S.\ Hale, Marie E.\ Rognes, Matthew W. Scroggs}\\
\end{flushright}
